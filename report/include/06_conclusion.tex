\section{Conclusion}\label{sec:conclusion_section}
This paper illustrated the implementation of the Multidimensional Scaling algorithm and the performances that can be obtained when it is used for estimating the location of a swarm of drones. Particularly, two different simulations were assessed to demonstrate its potential: a numerical one to collect data and to compute the average behaviour and the associated error, and a Software-in-the-loop one to demonstrate potential real-world applications. \par

The performance metrics were compared to a more classical but less robust algorithms known in the literature as trilateration. It has demonstrated that MDS always outperformed trilateration in all the different studied scenarios, in which different Gaussian-distributed noise was added to dynamics, measurement or clock-time. \par

The method studied in this work introduces for the first time the estimation of a swarm of fully-connected drones via infrastructure-free measurements, in which one node, designated as anchor, is responsible to move around and collect different measurements to make the algorithm run. \par

Future improvements and further studies can be conducted, such as the introduction of an extended-MDS or the integration of its covariance estimation to better update a Kalmann filter.